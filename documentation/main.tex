\documentclass[colorback,accentcolor=tud1c,11pt]{tudreport}
\usepackage[english]{babel}
\usepackage[utf8x]{inputenc}
%\usepackage[T1]{fontenc}

\usepackage{booktabs}
%\usepackage{multirow}
%\usepackage{longtable}
\usepackage{listings}
\usepackage{graphicx}
\usepackage{subfigure} 	
\usepackage{float}
\usepackage{amsmath}
\usepackage{rotating}
\usepackage{multirow}

\newcommand\todo[1]{\textcolor{red}{#1}}
\newcommand\code[1]{\texttt{#1}}
%\usepackage{floatflt}

\graphicspath{{./img/}}

%\newlength{\longtablewidth}
%\setlength{\longtablewidth}{0.675\linewidth}

\title{Mini-task report: SDC with simulated annealing}
\subtitle{Ludwig Meysel, Mitja Stachowiak}

\begin{document}
  \maketitle

  \chapter{Introduction}
  The task was to implement a simulated annealing approach (SA) for SDC (\underline{s}ystem of \underline{d}ifference \underline{c}onstraints). LPsolve is used to get a schedule for a given set of constraints. The SA-algorithm mutates the order of the constraints to reduce the number of clock cycles of the schedule.
  
  
  
  \chapter{Resources and Constraints}
  The constraints consist of data flow - and resource constraints. The data flow constraints determine, that no operation must start, before all predecessors have obtained a result. The resource constraints prevent, that the same resource is used twice at the same time.
  
  \section{Resources}
  Each hardware has a certain amount of resources and resource types. There is a fixed list of operations given in the framework:\\
  \begin{tabular}{ c | r | r }
  	Operation name (ar) & delay & weight \\
  	\hline
  	MEM   &  2 & 9.0 \\
  	ADD   &  1 & 1.0 \\
  	SUB   &  1 & 1.4 \\
  	MUL   &  4 & 2.3 \\
  	DIV   & 18 & 4.3 \\
  	SH    &  1 & 2.0 \\
  	AND   &  1 & 2.0 \\
  	OR    &  1 & 2.0 \\
  	CMP   &  1 & 2.1 \\
  	OTHER &  1 & 1.0 \\
  	SLACK &  1 & 0.0 \\
  \end{tabular}\\
  Each resource type can support multiple operations. For this project, the resource(types) are assumed to be overlap-free:
  $$\neg \exists R_1, R_2 \in Resources; Op_1, Op_2 \in Operations : Op_1 \in R_1 \land Op_1 \in R_2 \land Op_2 \in R_1 \land Op_2 \notin R_2$$
  Each resource can handle one operation within a certain time (delay). Multiple resources of the same type may exist.
  
  \section{Data Flow Constraints}
  The data flow constraints are fixed and only need to be computed once. \\
  
  \section{Resource Constraints}
  The data flow constraints are fixed and only need to be computed once. \\
  



  \chapter{Simulated Annealing}
  The principal structure of any simulated annealing looks like this:\\
  \fbox{\parbox{0.9\linewidth}{
  S = RandomConfiguration();\\
  T = InitialTemperature();\\
  while (ExitCriterion()==false) \{\\
  \phantom{}~~while (InnerLoopCriterion() == false) \{\\
  \phantom{}~~~~S\textsubscript{new} = Generate(S);\\
  \phantom{}~~~~$\Delta$ C = Cost(S\textsubscript{new})-Cost(S);\\
  \phantom{}~~~~r = random(0,1);\\
  \phantom{}~~~~if (r < $e^{- \Delta C/T}$) S = S\textsubscript{new}\\
  \phantom{}~~\}\\
  \phantom{}~~T = updateTemperature();\\
  \}
  }}\\ \\
  The implementation is located in scheduler/SASDC.java:schedule. The parameters are:
  \begin{itemize}
  	\item \emph{Random Configuration} ...
  	\item \emph{Initial Temperature} is determined by applying n(nodes) random changes and saving the costs of each change. T is then $20 * standardDeviation(costs)$.
  	\item \emph{Exit Criterion} is the condition, when the simulated annealing should stop. For each temperature, the number of applied changes and the number of accepted changes is counted. When less then 12\% of the changes are accepted, the algorithm stops.
  	\item \emph{Update Temperature} decreases T by a factor tu, which depends on the acceptance ratio as well:
  	\begin{tabular}{ c | c }
  		acceptance ratio (ar) & temperature factor (tu) \\
  		\hline
  		> 96\% & 0.5 \\
  		96 .. 80\% & 0.9 \\
  		80 .. 15\% & 0.95 \\
  		< 15\% & 0.8 \\
  	\end{tabular}
    \item \emph{Inner Loop Criterion} determines, how many changes are tested for the same temperature. Each change usually moves one node in the ordering of constraint-equations. The larger the number of nodes becomes, the more often each node should be moved, so the number of iterations should depend on the node count. Further more, there is a quality factor $\in [1 .. 10]$ for the algorithm, which can be passed via the third program argument. The formula $n_{inner} = \left\lceil quality * n_{nodes}^{4/3} \right\rceil$ is known to yield a result, thats quality belongs to the given quality.
  \end{itemize}


  \chapter{Evaluation}
  \vspace{100pt}
  \begin{tabular}{ c | r | r | r | r | r | r | r }
    File &
    \begin{rotate}{60} Number of Nodes \end{rotate} \hspace{3pt} &
    \begin{rotate}{60} cost fkt of ASAP \end{rotate} \hspace{10pt} &
    \begin{rotate}{60} cost fkt of ALAP \end{rotate} \hspace{10pt} &
    \begin{rotate}{60} cost fkt of SA/SDC \end{rotate} \hspace{10pt} &
    \begin{rotate}{60} Quality factor of SA \end{rotate} \hspace{3pt} &
    \begin{rotate}{60} Number of Iterations \end{rotate} \hspace{12pt} &
    \begin{rotate}{60} Runtime / s \end{rotate} \hspace{12pt} \\
   \hline
   \multirow{3}{*}{ADPCMn-decode-271-381} & \multirow{3}{*}{27} & \multirow{3}{*}{64.8} & \multirow{3}{*}{56.0} & 31.9 & 1 & 20494 & 29.39 \\
    &  &  &  & 30.5 & 5 & 82216 & 113.38 \\
    &  &  &  & 30.5 & 10 & 144181 & 185.55 \\
   \multirow{3}{*}{ADPCMn-decode-425-472} & \multirow{3}{*}{12} & \multirow{3}{*}{23.8} & \multirow{3}{*}{23.8} & 34.2 & 1 & 85 & 0.11 \\
    &  &  &  & 21.8 & 5 & 7591 & 6.37 \\
    &  &  &  & 21.8 & 10 & 11001 & 9.13 \\
   \multirow{3}{*}{ADPCMn-decode-524-553} & \multirow{3}{*}{7} & \multirow{3}{*}{25.4} & \multirow{3}{*}{26.4} & 25.4 & 1 & 15 & 0.02 \\
    &  &  &  & 25.4 & 5 & 68 & 0.07 \\
    &  &  &  & 25.4 & 10 & 135 & 0.13 \\
   \multirow{3}{*}{ADPCMn-decode-559-599} & \multirow{3}{*}{9} & \multirow{3}{*}{26.1} & \multirow{3}{*}{27.2} & 28.2 & 1 & 20 & 0.03 \\
    &  &  &  & 25.1 & 5 & 4137 & 3.26 \\
   &  &  &  & 28.2 & 10 & 377 & 0.35 \\
   \multirow{3}{*}{ADPCMn-decode-631-729} & \multirow{3}{*}{23} & \multirow{3}{*}{44.8} & \multirow{3}{*}{40.0} & 33.3 & 1 & 5017 & 5.81 \\
    &  &  &  & 24.3 & 5 & 13121 & 14.65 \\
    &  &  &  & 22.9 & 10 & 108076 & 120.56 \\
   \multirow{3}{*}{ADPCMn-decode-771-791} & \multirow{3}{*}{5} & \multirow{3}{*}{13.5} & \multirow{3}{*}{13.5} & 22.5 & 1 & 10 & 0.01 \\
    &  &  &  & 22.5 & 5 & 44 & 0.04 \\
    &  &  &  & 22.5 & 10 & 87 & 0.08 \\
   \multirow{3}{*}{ADPCMn-decode-803-832} & \multirow{3}{*}{8} & \multirow{3}{*}{18.8} & \multirow{3}{*}{18.8} & 20.2 & 1 & 17 & 0.02 \\
    &  &  &  & 20.2 & 5 & 81 & 0.08 \\
    &  &  &  & 20.2 & 10 & 161 & 0.16 \\
   \multirow{3}{*}{AESrkgcyclic} & \multirow{3}{*}{24} & \multirow{3}{*}{24.4} & \multirow{3}{*}{23.4} & 37.4 & 1 & 421 & 0.61 \\
    &  &  &  & 32.4 & 5 & 13881 & 18.84 \\
    &  &  &  & 30.4 & 10 & 28414 & 38.14 \\
   \multirow{3}{*}{BLAKE256Digest-processBlock-160-230} & \multirow{3}{*}{21} & \multirow{3}{*}{57.4} & \multirow{3}{*}{58.4} & 32.4 & 1 & 10905 & 12.11 \\
    &  &  &  & 22.4 & 5 & 58291 & 64.11 \\
    &  &  &  & 31.4 & 10 & 175161 & 194.06 \\
   \multirow{2}{*}{BLAKE256Digest-processBlock-189-1577} & \multirow{2}{*}{308} & \multirow{2}{*}{414.9} & \multirow{2}{*}{128.7} & 110.9 & 1 & 6244 & 241.63 \\
    &  &  &  & 97.9 & 5 & 31204 & 1236.87 \\
   \multirow{3}{*}{ContrastFilter-filter-13-252} & \multirow{3}{*}{47} & \multirow{3}{*}{72.0} & \multirow{3}{*}{67.7} & 50.8 & 1 & 4251 & 9.49 \\
   &  &  &  & 37.7 & 5 & 24622 & 54.00 \\
    &  &  &  & 33.2 & 10 & 101821 & 218.61 \\
   \multirow{1}{*}{ECOH256Digest-AES2RoundsAll-2-666} & \multirow{1}{*}{179} & \multirow{1}{*}{262.3} & \multirow{1}{*}{174.4} & 58.1 & 1 & 175567 & 2733.60 \\
  \end{tabular}\\
  The table above compares the results of simple ASAP / ALAP-Schedules with the results of the implemented simulated annealing algorithm.
 


 
 \chapter{Conclusion}
 

%  \bibliographystyle{plain}
%  \bibliography{references}
\end{document}

